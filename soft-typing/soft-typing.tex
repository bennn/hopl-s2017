\documentclass{article}
\usepackage[colorlinks=true]{hyperref}
\usepackage{amsmath}
\usepackage{amssymb}
\usepackage{mathpartir}
\usepackage{fourier-orns}
\usepackage{amsthm}
\usepackage{dashrule}
\usepackage{tikz}
\usetikzlibrary{shapes}
\usepackage{textgreek}
\usepackage{multicol}
\usepackage{listings}
%\usepackage[margin=1in]{geometry}
\usepackage{stmaryrd}

\mathchardef\mhyphen="2D

\renewcommand{\maketitle}[1]{{%
\flushleft\textsf{#1 \hfill \today} \\
\textsf{\hfill Ben Greenman} \\
\vspace{1mm}
\hrule
\vspace{0.4cm}}}


\newcommand{\elam}[2]{\lambda\,#1.\,#2}
\newcommand{\elet}[2]{\mathsf{let}~#1~\mathsf{in}~#2}
\newcommand{\tall}[2]{\forall\,#1.\,#2}
\newcommand{\trec}[2]{\mu\,#1.\,#2}
\newcommand{\tint}{\mathsf{Int}}
\newcommand{\tunit}{\mathsf{Unit}}
\newcommand{\eap}{\mathsf{ap}}
\newcommand{\ecap}{\mathsf{CHECK-ap}}
\newcommand{\eif}{\mathsf{if}}
\newcommand{\purescheme}{\mathsf{PureScheme}}
\newcommand{\tbool}{\mathsf{Bool}}
\newcommand{\coloneq}{::\,=}
\newcommand{\hdash}{{\vspace{1ex}}\hdashrule{\linewidth}{0.2ex}{1em}{\vspace{1ex}}}
\newcommand{\naturals}{\mathbb{N}}
\newcommand{\fplus}{\texttt{+}}
\newcommand{\fminus}{\mhyphen}

\newcommand{\localpdf}[2]{#2} % links to local pdfs not working, oh well
\newcommand{\figref}[1]{Figure~\ref{#1}}
\newcommand{\secref}[1]{Section~\ref{#1}}

\newtheorem*{theorem}{Theorem}
\newtheorem*{proposition}{Proposition}

\newcommand{\qa}[2]{\vspace{2ex}\noindent \textbf{Q.} #1 \newline \noindent \textbf{A.} #2}

\newcommand{\angles}[1]{\langle #1 \rangle}
\newcommand{\flang}{\mathsf{L}}
\newcommand{\fval}{\mathsf{V}}
\newcommand{\fden}[1]{\llbracket #1 \rrbracket}
\newcommand{\fwrong}{\mathsf{wrong}}
\newcommand{\etrue}{\mathsf{true}}
\newcommand{\efalse}{\mathsf{false}}
\newcommand{\ttrue}{\mathsf{True}}
\newcommand{\tfalse}{\mathsf{False}}
\newcommand{\tnil}{\mathsf{Nil}}
\newcommand{\tcons}{\mathsf{Cons}}
\newcommand{\enil}{\mathsf{nil}}
\newcommand{\eadd}{\mathsf{add1}}
\newcommand{\ecadd}{\mathsf{CHECK{\mhyphen}add1}}
\newcommand{\echecked}{\mathsf{checked}}
\newcommand{\eintp}{\mathsf{integer?}}
\newcommand{\econs}{\mathsf{cons}}
\newcommand{\efirst}{\mathsf{first}}
\newcommand{\ttypeof}{\emph{TypeOf}}
\newcommand{\eempty}{\emph{Empty}}
\newcommand{\tsofttypeof}{\emph{SoftTypeOf}}
\newcommand{\eclose}{\emph{Close}}
\newcommand{\esoftclose}{\emph{SoftClose}}
\newcommand{\esecond}{\mathsf{second}}
\newcommand{\absent}[1]{\overline{#1}}

\begin{document}
\maketitle{Soft Typing}

\newcommand{\todo}[1]{\textbf{TODO: #1}}

\begin{abstract}
Lecture notes, April 14th, \href{http://www.ccs.neu.edu/home/matthias/7480-s17/index.html}{HOPL 2017}.
Don't mind the ornaments.
\end{abstract}

\subsection*{Soft Typing: Philosophy and Motivation}

\begin{quote}
``\ldots the computer \ldots is willing to accept almost any sequence of
  instructions and make sense of them at its own level.
  That is the secret of the power, flexibility, and even reliability of computer
  hardware, and should therefore be cherished.'' --- Tony Hoare~\cite{h-dtic-1973}
\end{quote}

% TODO change margins on these lists, or on the text. Looks bad
\noindent \aldine~Static type systems have well-known benefits.
\begin{enumerate}
\item
  A static type system can catch programming errors early (typos, logical mistakes),
  and exhaustively\textemdash by exploring all paths through the program.
\item
  Type signatures are a useful form of documentation.
\item
  Optimizing compilers can use type information to compile more efficient
  code.\footnote{How useful is compiling with types? This was the subject of two HOPL talks in March, see:
                 \url{https://github.com/nuprl/hopl-s2017/blob/master/lecture_notes/2017-03-24.md} and
                 \url{https://github.com/nuprl/hopl-s2017/blob/master/lecture_notes/2017-03-28.md}.}
\item
  Designing types first can help guide the design of programs.
\end{enumerate}

\noindent \bomb~Dynamic typing also has well-known benefits.
\begin{enumerate}
\item
  Unlike untyped languages, dynamically typed languages guarantee safety from
  memory errors (e.g. segfaults).
\item
  A dynamically typed language will run any syntactically correct program.
\item
  (consequence of point 2) Dynamically typed functions can be re-used on
  a larger class of arguments.
\item
  Dynamic typing is conceptually simpler than static typing; a programmer
  only needs to understand the runtime behavior of programs, not the behavior
  of the type checker.
\end{enumerate}

\noindent \textxswdown~\textbf{Note} to emphasize point 3 above, here is a program by
Alan Mycroft that cannot be typed in the ML type system~\cite{m-icop-1984}.

% TODO lstlisting
\begin{verbatim}
     let rec f(x : structure) = case x of
       (basecase(y): ...
       |listcase(y): g(y, (hd, tl, null))
       |dlistcase(y): g(y, (dhd, dtl, dnull)))
     and g(x : a, (xhd: a->b, xtl: a->a, xnull: a->bool)) =
       if xnull(x)
       then ()
       else (f(xhd x), g(xtl x, (xhd, xtl, xnull)))
\end{verbatim}

Mycroft wrote this program while developing the ML compiler.~\textbf{End Note.}

\noindent \danger~Why not combine all the benefits in a single system?
The dream would be, an ideally-flexible type system that can type check any
dynamically-typed program.
\begin{itemize}
\item
  Standard typecheckers try to prove that a program is ``good'', for some
   notion of ``good'' (think: \emph{will not segfault at runtime}).
  The set of ``good'' programs is almost always recursively enumerable,
   therefore a \emph{decidable} type checker \textbf{must reject some ``good'' programs}.
\item
  Dynamically typed programs tend to rely heavily on untagged unions, recursive
   types, and structural subtyping.
  These features make type inference and type checking more difficult.
  In particular, Curry-style typing assumes that datatypes are disjoint.

  \emph{Philosophy:} the types in a dynamically-typed program come from the
  \emph{data} in that program.
  Functions rely heavily on structural subtyping, guided by dynamic type tests.
\end{itemize}

%% TODO more citations
One approach to solving this problem is to design a new type system~\cite{m-icop-1984},
 in the hope that the system can accomodate enough programming idioms to be
 widely useful.

A second approach is soft typing.
Given a type environment $\Gamma$ and expression $e$, a soft type checker
 infers type information in $e$ and rewrites the program to a
 semantically-equivalent and type-safe program $e'$, which has static type $\tau$.

$$ \Gamma \vdash e \Rightarrow e' : \tau $$

Many more details follow.


\subsection*{Design Criteria for a Soft Type Checker}

Fagan's dissertation outlines the design requirements for a soft type checker.
The following list is adapted from Section 1.1 of the dissertation~\cite{f-thesis-1991}.
\begin{enumerate}
\item
  No syntactically correct programs are excluded (i.e. rejected by the type checker).
\item
  Run-time safety is assured by run-time checks if such checks cannot be
  safely eliminated.
\item
  The type checking process must be unobtrusive, where \emph{unobtrusive} is
  characterized by two principles:
\subitem \emph{Minimal text principle}
  The type checking system should function in the absence of
  programmer-supplied type declarations.
\subitem \emph{Minimal failure principle}
  The checker must pass a ``large fraction'' of dynamic programs that will not
  produce an execution error.
\end{enumerate}

%% TODO format, somehow
For example,
\begin{verbatim}
     (+ 2 2)
     ;; no casts

     (+ 2 (lambda (x) x))
     ;; add cast

     (if #true 2 2)
     ;; well-typed

     (if #true 2 (lambda (x) x))
     ;; well-typed, no casts

     (define (flatten tree)
       (cond
        [(null? tree)
         '()]
        [(pair? tree)
         (append (flatten (first tree))
                 (flatten (rest tree)))]
        [else ;; tree is a list
         (list tree)]))
     ;; well-typed, no casts
\end{verbatim}

\begin{quote}
  \emph{Any questions on the design criteria? These are important in subtle ways later.}
\end{quote}


\subsection*{Inferring Types from Untyped Code}

Given the design criteria, the obvious challenge is how to infer descriptive
 types for annotation-free code.
Hindley-Milner type inference works well for a similar problem in ML programs,
 so it is a natural place to start looking for a solution.


\subsubsection*{Hindley-Milner Type Inference}

Types $\tau$ are disjoint, terms $e$ are the lambda calculus.
Type schemes $\sigma$ quantify over type variables $\alpha$.


\[\begin{array}{l c l}
  \tau & \coloneq & \alpha \mid \tint \mid \tbool \mid \tau \rightarrow \tau
\\ e   & \coloneq & x \mid \elam{x}{e} \mid e~e \mid \elet{x = e}{e}
\\\sigma & \coloneq & \tau \mid \tall{\alpha}{\sigma}
\\\Gamma & \coloneq & \cdot \mid x\!:\!\sigma,\Gamma
\\ S & \coloneq & \cdot \mid \alpha\!:\!\sigma,\Gamma
  \end{array}\]

% TODO accent on Damas name
Milner and Damas give syntax-directed type checking rules.
Reading these rules bottom-up and using fresh type variables each time a new
 type is required yields an efficient inference algorithm (Algorithm W)
 parameterized by a solver.

Type Checking: \fbox{$\Gamma \vdash e : \tau$}
\qquad\qquad
Type Inference: \fbox{$W(\Gamma, e) = (S, \tau)$}

\noindent Algorithm W has useful properties:

\begin{theorem}
  (soundness) If $W(\Gamma, e) = (S, \tau)$, then $\Gamma\circ S \vdash e : \tau$
\end{theorem}

\begin{theorem}
  (completeness) If $\Gamma\circ S_0 \vdash e : \tau_0$ then $W(\Gamma, e) =
  (S_1, \tau_1)$ and there exists a substitution $S_2$ such that $\Gamma \circ
  S_0 = \Gamma \circ S_1 \circ S_2$ and $\tau_0 = \tau_1 \circ S_2$ modulo
  instantiation of quantified type variables in $\tau_0$
\end{theorem}

\begin{theorem}
  (principal types) If $W(A,e) = (S,\tau)$ then the generalization
  of $\tau$ under $\Gamma \circ S$ is a principal type of $e$ under $\Gamma \circ S$.
\end{theorem}

To summarize, the key ideas with Hindley-Milner are to generate
 \emph{type equality constraints} from the syntax of the program, then
 \emph{solve} the type constraints for a most general unifier
 (using Robinson Resolution).


\subsubsection*{Adapting Hindley-Milner}

A non-starter idea is to directly apply the Hindley-Milner rules to dynamically
 typed programs.
This is going to fail for many programs; Fagan gives a tautology checker as an
 example; the {\tt flatten} program from above is another good example:

\begin{verbatim}
  let rec tautology f =
    if f == true then true
    else if f == false then false
    else (tautology (f true)) and (tautology (f false))
\end{verbatim}

At a minumum, the type system needs \emph{recursive types} and \emph{untagged unions}.
%% TODO cites
The recursive types are not so difficult, but untagged unions mean that datatypes
 are no longer disjoint.
In particular, non-disjointness means the type system needs a subtyping judgment;
 the rule for typing a function application must look like:
% TODO color the subtyping part

\begin{mathpar}
  \inferrule*{
    \Gamma \vdash f : \tau \rightarrow \tau'
    \\
    \Gamma \vdash e : \tau_e
    \\
    \tau_e \subseteq \tau
  }{
    \Gamma \vdash f~e : \tau'
  }
\end{mathpar}

\noindent \textbf{Challenge 1}: how to turn $\tau_e \subseteq \tau$ into a constraint?
There are two ``obvious'' solutions (for some value of obvious).

\begin{itemize}
\item
  \emph{(retained inequalities)}
  Do nothing!
  Keep $\tau_e \subseteq \tau$ as a constraint, and change the solver from
   Robinson resolution to something more sophisticated, that can handle
   subset constraints.
\item
  \emph{(convert to equalities)}
  Convert the inequality to an equality using the venerable technique of
   \emph{slack variables}.
  In this case, $\tau_e \subseteq \tau$ generates the constraint
   $\tau_e \cup \alpha = \tau$.
\end{itemize}

\noindent \lefthand~Fagan chooses to \emph{convert} the inequalities.

\noindent \textbf{Note}: his choice is partially motivated by three perceived
 challenges with retained inequalities:
\begin{itemize}
\item
  the set of inequalities will grow quickly, may not be possible to simplify, and may require exponential time to solve;
\item
  it is unclear how to convert an inconsistent set of constraints into a safety-ensuring runtime check;
\item
  existing type systems with retained inequalities do not handle parametric polymorphism.
\end{itemize}

We will revisit these issues later, in the section on \emph{Soft Typing with
 Conditional Types}.
\textbf{End Note}
% TODO use proper section & label/ref ????

\noindent \textbf{Challenge 2}: the Hindley-Milner solver (Robinson resolution)
 finds most-general-unifiers only for terms in a \emph{free} term algebra.
But we have terms like $\tau_e \cup \alpha = \tau$ that use the union operator ($\cup$).
Union is not free!
It is associative, commutative, and idempotent (ACI).

\noindent \emph{What to do?}
We can start like computer science normally does, by asking mathematicians for
 help.
And it turns out, general unification theory has results for ACI unification.
Unfortunately ACI unificiation:
\begin{itemize}
\item
  produces multiple unifiers;
\item
  does not support the circularity needed to infer recursive types;\footnote{This may have changed, but was true in 1990.}
\item
  does not offer any distributivity, but we need equations like $c~\tau \cup c~\tau' = c(\tau \cup \tau')$
  (furthermore, Fagan notes that distributive, associative theories do not have unification methods).
\end{itemize}


\noindent \emph{What to do (2)?}
Fagan restricts the solution space to \emph{discriminative} unions over rational
 regular trees.
\begin{itemize}
\item
  A \emph{discriminative} union contains at most one occurrence of each type constructor.
  This solves the ACI issues because (AC) we can sort the constructors (I) and
  idempotence becomes a non-issue by definition.
\item
  A \emph{rational regular tree} is a possibly-infinite tree with a finite set
  of non-isomorphic subtrees. (Example: reduction graph for $(K~I~\Omega)$.)
\end{itemize}

\begin{proposition}
Algorithm W can infer a substitution for two rational
 regular trees, if a substitution exists.
\end{proposition}

\begin{proposition}
There is a useful type system for dynamically typed programs whose types
 are isomorphic to rational regular trees.
\end{proposition}


%% -----------------------------------------------------------------------------
\subsection*{Diversion: The R\'{e}my Encoding}

% TODO cite
Not going to prove those propositions, but want to give a general idea of how
 the encoding works.
It is based on a technique by Didier R\'{e}my for adding records with structural
 subtyping to ML.


\subsubsection*{Setup}

Start with the ML language, want to add record types.
A record type is a sequence of field labels $l$ and types.

\[\begin{array}{l c l}
  e & \coloneq & \ldots \mid \{ l = e , \ldots \}
\\\tau & \coloneq & \ldots \mid \{ l : \tau , \ldots \}
  \end{array}\]

Also want structural subtyping, so if we have a function that extracts the
value of the {\tt left} label on a record, it works for records with more
labels.

\begin{verbatim}
    let get_left (x : {left : 'a}) : 'a =
      x.left

    get_left {left = 1};;
    get_left {left = 1, right = 2};;
\end{verbatim}


\subsubsection*{Insight and Solution Sketch}

Suppose the set $l$ of labels has only 2 elements and records can only contain
 values of type $\tunit$.
Then we can encode all possible record types in a 4-element lattice, where
 each label is either present or absent in each type.

%% TODO diagram lattice

For example, $X = {\tt xo}$ and $Y = {\tt xx}$ are two possible record values.
Their types are obvious.

One way to add subtyping would be to pretend that $X$ and $Y$ denote multiple values:

\[
  X = {\tt oo} \mid {\tt xo}
\\
  Y = {\tt oo} \mid {\tt xo} \mid {\tt ox} \mid {\tt xx}
\]

Then we can take the union or intersection of these ``value sets''.
But we do not want multiple values; we want one value for $X$ and one value for $Y$.

R\'{e}my's solution is to encode the various types of $X$ and $Y$ with polymorphism.
He does so in two steps:
\begin{enumerate}
\item
  Add \emph{flags} to the labels in record types, indicating whether the
  label is present ({\tt +}), absent ({\tt +}), or unknown.
\item
  Introduce variables $\phi$ that range over the flags.
\end{enumerate}

Using this encoding, the final types for $X$ and $Y$ are:

\[
  X : \{\phi_0 : \tunit, {\tt -} : \tunit\}
  \\
  Y : \{\phi_1 : \tunit, \phi_2 : \tunit\}
\]

The absent field in $X$ has flag {\tt -}.
The present fields in $X$ and $Y$ have a flag variable, meaning they could be
 used or forgotten depending on the context.
For example, the expression {\tt if e then X else Y} has type $\{\phi : \tunit, \texttt{-} : \tunit\}$
by unifying $\phi_2$ with {\tt -}.
Also, the type for {\tt get\_left} is $\{left : \texttt{+}, right : \phi \} \rightarrow \tunit$.

If you have more labels, just make longer types.
That's the essence of the encoding.

\begin{quote}
  ``Variants are to concrete data types what records are to labelled products.'' --- R\'{e}my
\end{quote}

Adding structural subtyping for variants is straightforward; a variant type is
 the sum of all appropriately-flagged possibilities.

R\'{e}my adds recursive types, records, and variants to the ML type system and
 proves the same \emph{soundness}, \emph{completeness}, and \emph{principle types}
 properties as Milner.
Clean and simple.


%% -----------------------------------------------------------------------------
\subsection*{Fagan's Soft Type System}

Back to soft typing, ``the'' type for expressions in a dynamically typed program
 is one large variant of all the possible base types.

Given the following grammar of types,

\[\begin{array}{l c l}
  \tau & \coloneq & \tint \mid \tbool \mid \tau \rightarrow \tau
  \end{array}\]

we can express:
\begin{itemize}
\item
  Integers $: [\tint +, \tbool -, \rightarrow -, \alpha_0, \alpha_1]$
\item
  Booleans or Integers $: [\tint +, \tbool +, \rightarrow -, \alpha_0, \alpha_1]$
\item
  Functions from integers to booleans $: [\tint -, \tbool -, \rightarrow +, [\tint +, \tbool -, \rightarrow -, \alpha_0, \alpha_1], [\tint -, \tbool +, \rightarrow -, \alpha_2, \alpha_3]]$
\end{itemize}

These are types written in Fagan's notation.
The type variables cover ``stucturally similar'' positions.
Wright makes further use of type variables, so we will use his system instead.

% Wrap-up
% - types->vectors -> infer -> types (details in paper)
% - lots of example derivations (in paper)
% - insert narrowers to preserve safety and semantics of possibly-unsafe programs


%% -----------------------------------------------------------------------------
\subsection*{Wright's Soft Type System}

Wright presents a soft type system for $\purescheme{}$.

\subsubsection*{Expressions, Operational Semantics, Stuck Programs}
Terms $e$ are lambda terms, and are made of values $v$, primitives $c$, base
 values $b$, and primitive functions $p$.

\[\begin{array}{l c l}
   e & \coloneq & v \mid (\eap~e~e) \mid (\ecap~e~e) \mid (\eif~e~e~e) \mid (\elet{(x~e)}{e})
\\ v & \coloneq & c \mid x \mid (\elam{x}{e})
\\ c & \coloneq & b \mid p
\\ b & \coloneq & \mathsf{integer} \mid \etrue{} \mid \efalse{} \mid \enil{}
\\ p & \coloneq & \eadd \mid \econs \mid \efirst \mid \esecond \mid \eintp \mid \ecadd \mid \ldots
  \end{array}\]

The unconventional thing about $\purescheme{}$ is that applications are labeled
 as unsafe ($\ecap$) or checked ($\eap$).
Primitive operations also come in unsafe (e.g. $\eadd$) and checked flavors ($\ecadd$).
If the checks are present, the operational semantics will raise a special error
 $\echecked$ instead of undefined behavior.

\[\begin{array}{l c l l}
   E[(\eap~(\elam{x}{e})~v)] & \mapsto & E[e[x/v]] &
\\ E[(\ecap~(\elam{x}{e})~v)] & \mapsto & E[e[x/v]] &
\\ E[(\elet{x~v}{e})] & \mapsto & E[e[x/v]] &
\\ E[(\eif~v~e_0~e_1)] & \mapsto & E[e_0] & \mbox{ when } v \neq \efalse
\\ E[(\eif~\efalse~e_0~e_1)] & \mapsto & E[e_1] &
\\ E[(\eap~p~v)] & \mapsto & E[v]  & \mbox{ when } v' = \delta(p, v)
\\ E[(\eap~p~v)] & \mapsto & \echecked & \mbox{ when } \echecked = \delta(p, v)
\\ E[(\ecap~p~v)] & \mapsto & E[v'] & \mbox{ when } v' = \delta(p, v)
\\ E[(\ecap~p~v)] & \mapsto & \echecked & \mbox{ when } \echecked = \delta(p, v)
\\ E[(\ecap~b~v)] & \mapsto & \echecked &
  \end{array}\]

The $\delta$ metafunction what you would expect.
Here are its cases for checked primitives:

\[\begin{array}{l c l l}
  \delta(\mathsf{CHECK{\mhyphen}p}, v) & = & v' & \mbox{ when } v' = \delta(p, v)
\\\delta(\mathsf{CHECK{\mhyphen}p}, v) & = & \echecked & \mbox{ when } \delta(p, v) \mbox{ is undefined}
  \end{array}\]

The purpose of showing these definitions was to define what it means for a
 program to be stuck.
Stuck evaluation contexts have the following forms:

\[\begin{array}{l l}
   E[(\eap~p~v)] & \mbox{ where } \delta(p, v) \mbox{ is undefined}
\\ E[(\ecap~p~v)] & \mbox{ where } \delta(p, v) \mbox{ is undefined}
\\ E[(\eap~b~v)]
  \end{array}\]

\begin{proposition}
(dynamic typing) terms $e$ that always use $\ecap$ and checked primitives never
 transition to a stuck state via the reflexive, transitive closure of $\mapsto$
\end{proposition}

The (upcoming) soft type system will justify replacing some checked operations
 with unsafe ones.


\subsubsection*{Type System}

%% TODO fill in here
First static types.
Second, soft types.

The static type system will have the property SOUNDNESS.

The soft type system will have the property TODO.


Wright's grammar for static types $\tau$ is unconventional.
To start, a type $\tau$ can be empty ($\emptyset$) or a type variable ($\alpha$)
 or a recursive type ($\trec{\alpha}{\tau}$).
A type can also be a sequence of tagged ($k$), flagged ($f$), parameterized
 types, followed by a type variable or empty type.
A tag is a type constructor; each constructor has a fixed arity.
The grammar for flags is the same used by R\'{e}my~\cite{r-popl-1989}.

Type schemes $\Sigma$ bind type variables.
Substitutions $S$ are used in unification.

\[\begin{array}{l c l}
   G0 & \coloneq & \alpha \mid \emptyset
\\ G1 & \coloneq & (k^f~\tau~\ldots) \cup G1 \mid G0
\\ \tau & \coloneq & G0 \mid G1 \mid \trec{\alpha}{\tau}
\\ k & \coloneq & \tint \mid \ttrue \mid \tfalse \mid \tnil \mid \rightarrow
\\ f & \coloneq & + \mid \mhyphen \mid \phi
\\ \Sigma & \coloneq & \tall{\nu}{\Sigma} \mid \tau
\\ S & \coloneq & \cdot \mid \alpha = \Sigma,S \mid \phi = f, S
\\ \nu & \coloneq & \alpha \mid \phi
  \end{array}\]

\noindent \leafNE~\textbf{Note:} the above grammar is incomplete,
 when we get to soft types we will need \emph{absent} type and flag variables,
  along with absent types $\absent{\tau}$ and absent flags $\absent{f}$.
 Will clarify their purpose later, but for now the important point is that
  the full grammar for substitutions $S$ maps absent variables to absent non-terminals:

 \[\begin{array}{l c l}
    S & \coloneq & \cdot \mid \alpha = \Sigma,S \mid \phi = f, S \mid \absent{\phi} = \absent{f}, S \mid \absent{\alpha} = \absent{\tau}, S
 \end{array}\]

Examples:
\begin{itemize}
\item
  $4 : (\tint +) \cup \emptyset$
\item
  $4 : (\tint +) \cup (\ttrue \mhyphen) \cup \emptyset$
\item
  $\etrue : (\ttrue +) \cup (\tfalse +) \cup \emptyset$
\item
  $\eintp : (\rightarrow + \alpha ((\ttrue +) \cup (\tfalse +) \cup \emptyset)) \cup \emptyset$
\end{itemize}

(A little strange, but not as difficult to write as Fagan's.)

These types must also be discriminative; each constructor $k$ can appear at most
 once in a union type.
This means that type variables at the end of a union quantify \emph{only} over
 missing constructors.
Discriminativity also lets Wright re-order constructors in a union to a canonical order.

Example type scheme:

$$ \tall{\phi_0, \phi_1}{(\tint \phi_0) \cup (\tnil \phi_1) \cup \emptyset} $$

this scheme represents a lattice of four types, just like the R\'{e}my-encoded
 type for records with two labels.

Types for primitive operations are next on the agenda.
But you should keep in mind this general intuition:
\begin{itemize}
\item
  flag variables $\phi$ in a function domain allow subtyping at call sites
\item
  type variables $\alpha$ in a function codomain let the function return a \emph{supertype}
\end{itemize}

As a simple example illustrating the second point, consider this $\eif$-statement:

$$ \eif~e~1~\efalse $$

Its type should be the union of the types for the constants 1 and $\efalse$.
To make this union defined, their types are:

$$ 1 : \tall{\alpha}{(\tint +) \cup \alpha}
   \\
   \efalse : \tall{\alpha}{(\tfalse +) \cup \alpha}$$

Hence the union is:

$$ \tall{\alpha}{(\tint +) \cup (\tfalse +) \cup \alpha} $$

\noindent \textbf{Static Primop Types}

yolo

\noindent \textbf{Static Type Checking}

\noindent \textbf{Soft Primop Types}

\noindent \textbf{Soft Type Checking}

\noindent \textbf{Correctness}


\subsection*{Soft Typing for Scheme (R4RS)}

\textbf{Experience and Performance}



%% -----------------------------------------------------------------------------
\subsection*{Soft Typing with Conditional Types}
\label{sec:awl-popl-1994}

% Recall the issues from Fagan:
% - large set of inequalities
% - how to insert runtime checks?
% - 
% Lemma 7.1: where/how to insert inequalities

%% -----------------------------------------------------------------------------
\subsection*{Quasi-Static Typing}
\label{sec:t-popl-1990}

%% -----------------------------------------------------------------------------
\subsection*{Global Tagging Optimization by Type Inference}
\label{sec:h-lfp-1992}

%% -----------------------------------------------------------------------------
\subsection*{Questions}

In other words, these are things that do not fit in the real presentation, but
are interesting and might come up as questions.

\qa{
  Fundamental Theorem of Static Typing?
}{
  Fagan states and proves the ``fundamental theorem'', that a static type system
  must reject some meaningful programs~\cite{f-thesis-1991}.
  Here are the relevant definitions from Fagan:

  Let a \emph{programming language} be a triple $\angles{\flang, \fval, \fden{\cdot}}$
  where $\flang$ is the syntax of the language, $\fval$ are the value forms,
  and $\fden{\cdot}$ is a denotational semantics mapping syntax to values.

  A programming language is \emph{interesting} if (1) all terms have a value
  (2) the set $\fval$ includes values denoting errors and non-terminating
  computations (3) the language includes an $\mathsf{if}$-statement (4) the
  set of terminating programs is recursively enumerable but not
  recursive.\footnote{A \emph{recursively enumerable} set is a set that can be
  computed by a (partial) Turing machine. A \emph{recursive} set can be
  computed by a total Turing machine. The difference is roughly ``computable''
  vs. ``decidable''.}

  A \emph{good program} written in a language $\angles{\flang, \fval, \fden{\cdot}}$
  is a member of the set $\{e \mid e \in \flang \wedge \fden{e} \neq \fwrong \}$.

  A \emph{static type system} for a language $\angles{\flang, \fval, \fden{\cdot}}$
  computes a recursive set $\flang_W \subset \flang$ such that for all $e \in \flang$,
  the value $\fden{e}$ is not $\fwrong$.

  Fagan then proves:
  \begin{itemize}
  \item \textbf{Lemma 1.1} \emph{the set of good programs is not recursive}
  \item \textbf{Fundamental Theorem of Static Typing} \emph{any set of well-typed
   programs $\flang_W$ is a strict subset of the set of good programs}
  \end{itemize}

  The proofs are in Fagan's dissertation~\cite{f-thesis-1991}, but proving them is a nice exercise.
}

\qa{
  What were Wright's extensions, how did they work?
}{
  
}


%% -----------------------------------------------------------------------------
\subsection*{Appendix}

%% TODO hindley-milner inference rules
%% robinson resolution
% MGU(B,B) = ()
% MGU(-> A B, -> C D) = let S1 = MGU(A,C) in let S2 = MGU(S1 B, S1 D) in S1 union S2
% MGU(a, T) = (a -> T)
% MGU(T, a) = (a -> T)
% MGU(- -) = error

\footnotesize
\bibliographystyle{plain}
\bibliography{soft-typing}
\end{document}
